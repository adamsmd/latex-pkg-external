\RequirePackage{external}
\externalCode{example-catcode-standalone}{
  \DeclareFontFamily{U}{fselch}{}
  % We use `\scantokens` because \DeclareFontShape plays with the
  % catcodes and this code is being passed as the argument of another
  % command.
  %
  % The `\relax` at the end ensures that `\scantokens` does not
  % insert an extra space at the end.
  % See https://tex.stackexchange.com/questions/117906/use-of-everyeof-and-endlinechar-with-scantokens
  \scantokens{\DeclareFontShape{U}{fselch}{m}{n}{<-> s * [13] fselch10}{}\relax}
  \usepackage{pifont}}{%
  \Pisymbol{fselch}{1}}
