\documentclass{article}

\usepackage{external}

\usepackage{tcolorbox}
\tcbuselibrary{listingsutf8}
%\tcbset{sidebyside=true}

\def\RULE{\rule{0.2em}{7pt}\llap{\rule[8pt]{0.2em}{2pt}}}

\begin{document}

\section{Introduction}

\section{Examples}

\begin{tcblisting}{}
\RULE%
\external{example-inline-together}{\usepackage{amsmath}}{$\int$}%
\RULE
\end{tcblisting}

% NOTE: If you want to include display math in `\external` use the incantation: \[ \external{$\displaystyle ...$} \]


These examples use guide lines like the following (TODO):

\begin{tcblisting}{}
\RULE M\RULE
\end{tcblisting}

First two are blank

\begin{tcblisting}{}
\RULE%
\ExternalWrite{example-inline-separate}%
  {\usepackage{amsmath}}%
  {$\int$}%
\RULE

\RULE\ExternalRun{example-inline-separate}\RULE

\RULE\ExternalRead{example-inline-separate}\RULE
\end{tcblisting}

\section{Standalone}

First is blank

\newcommand{\showfile}[1]{%
  \begin{tcolorbox}[title=\texttt{#1}]
  \verbatiminput{#1}
  \end{tcolorbox}%
}

\showfile{example-standalone-simple.tex}

Note: be careful if you rename such a file, as you will also want to change the argument to \verb|ExternalCode|.

\begin{tcblisting}{}
\RULE\ExternalRun{example-standalone-simple}\RULE

\RULE\ExternalRead{example-standalone-simple}\RULE
\end{tcblisting}

\section{New commands}

TODO:
\newcounter{external}[table]

\begin{tcblisting}{}
\newexternal{\ext}{\stepcounter{external}}{\usepackage{amsmath}}

\RULE\ext{example-inline-new-\arabic{external}}{$\int$}\RULE

\RULE\ext{example-inline-new-\arabic{external}}{$\iint$}\RULE
\end{tcblisting}

% TODO: \newexternal*
\begin{tcblisting}{}
\newexternal*{\extS}{\stepcounter{external}}%
  {example-inline-new-\arabic{external}}%
  {\usepackage{amsmath}}

\RULE\extS{$\int$}\RULE

\RULE\extS{$\iint$}\RULE
\end{tcblisting}

\section{Problems}

\subsection{Hashes}

% Known problem: #1
% Known workaround: ##1

\begin{tcblisting}{}
\external{example-inline-hash}%
  {\def\foo#1{abc#1def}}%
  {\foo{xyz}}
\end{tcblisting}

Note double hash if standalone

\showfile{example-standalone-hash.tex}

\begin{tcblisting}{}
\ExternalRun{example-standalone-hash}
\ExternalRead{example-standalone-hash}
\end{tcblisting}

\subsection{catcodes}

% Known problem: catcode in body (e.g., \DeclareFont...)
% Known workarounds: \scantokens

% We use `\scantokens` because \DeclareFontShape plays with the
% catcodes and this code is being passed as the argument of another
% command.
%
% The `\relax` at the end ensures that `\scantokens` does not
% insert an extra space at the end.
% See https://tex.stackexchange.com/questions/117906/use-of-everyeof-and-endlinechar-with-scantokens

\begin{tcblisting}{}
\external{example-inline-catcode}{
  \DeclareFontFamily{U}{fselch}{}
  \scantokens{%
    \DeclareFontShape{U}{fselch}{m}{n}%
      {<-> s * [13] fselch10}{}\relax}
  \usepackage{pifont}}{%
  \Pisymbol{fselch}{1}}
\end{tcblisting}

\showfile{example-standalone-catcode.tex}
\begin{tcblisting}{}
\ExternalRun{example-standalone-catcode}
\ExternalRead{example-standalone-catcode}
\end{tcblisting}

\end{document}
